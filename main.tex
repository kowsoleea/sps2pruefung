\documentclass[10pt,a4paper,twoside]{report}
\usepackage{graphicx}
\usepackage[ngerman]{babel}
\usepackage{fancyhdr}
\usepackage{leading}
\usepackage{fontspec}
\usepackage[hidelinks]{hyperref}
\usepackage{amsmath}

\defaultfontfeatures{Ligatures=TeX}
\setmainfont{Garamond}
\setsansfont{Agfa Rotis Semisans}
\setmonofont{Consolas}

\textwidth 12cm
\marginparwidth 3cm
\marginparsep 8mm
\parindent 0cm
\parskip 2pt


%pagestyle
\pagestyle{fancy}
\renewcommand{\chaptermark}[1]{\markboth{#1}{} }
\renewcommand{\sectionmark}[1]{\markright{#1}{} }
\fancyhf{}
\fancyhead[LE,RO]{\thepage}
\fancyhead[LO,RE]{\nouppercase \leftmark}
\fancyheadoffset[LE,RO]{38mm}

\fancypagestyle{plain}{
	\fancyhf{}
	\fancyfoot[CE, CO]{\thepage}
	\renewcommand{\headrulewidth}{0pt}
}



\author{Ivanildo Kowsoleea\\Fachakademie für Sozialpädagogik Mühldorf}
\title{{\huge Musikalische Kalligrafie}\\
{\small oder: wie schreibe ich Noten richtig }}
\date{im Januar 2014}

\newcommand{\comment}[1]{\marginpar{\begin{flushleft}
            \textsf{#1}
        \end{flushleft}}}

\newcommand{\image}[4]{
	\begin{figure}[!ht]
		\centering
		\includegraphics[width=#4cm]{#1}
		\caption{#2}
		\label{#3}
	\end{figure}
}

\begin{document}
\maketitle
\tableofcontents

\leading{12pt}
\chapter{Einzelteile}

\section{Der Schlüssel}

Zuerst befassen wir uns mit dem Notenschlüssel. Es gibt drei Arten,
die G-Schlüssel, die F-Schlüssel und die C-Schlüssel
(siehe \autoref{notenschl}).
\image{img/schluessel.png}{Der G-Schlüssel, der F-Schlüssel und der C-Schlüssel}{notenschl}{8}

Der G-Schlüssel zeigt an, wo auf den Notensystem sich die Note $g^1$
befindet, genau da, wo die schleife in der Mitte anfängt. 
In Prinzip kann der G-Schlüssel auf jede Linie anfangen, heutzutage
jedoch fängt er auf der zweiten Linie von unten an.
In diesem Fall heißt der Schlüssel \emph{Violinschlüssel.}\comment{Violinschlüssel}

Der F-Schlüssel markiert die Note $f$ und der C-Schlüssel die Note $c^1$. Der F-Schlüssel auf der vierte Linie von unten nennt man auch \emph{Bassschlüssel.} Der C-Schlüssel auf der dritte Linie ist der \emph{Altschlüssel.}

Für uns ist nur der Violinschlüssel (siehe \autoref{violinschl}) von Interesse.
\image{img/violinschl.png}{Der Violinschlüssel}{violinschl}{3}
Der Violinschlüssel fangt auf der zweiten Linie von unten an, hoch bis zur dritten 
Linie, zurück zur untersten Linie, linksherum nach oben, außerhalb des Notensystems nach 
links, und mittig nach unten.

\section{Die Notenwerte}
Der Wert einer Note, der Notenwert, gibt an, wie lange eine Note dauert.
\comment{Notenwert = Dauer}
Die Notenwerte die hier behandelt werden sind: die ganze Note, die halbe Note,
die Viertelnote, die Achtelnote und die Sechzehntelnote. In \autoref{nwerte}
sind diese Notenwerte einmal dargestellt.
\image{img/nwerte.png}{von links nach rechts: die ganze Note, die halbe Note,
die Viertelnote, die Achtelnote und die Sechzehntelnote}{nwerte}{8}


\section{Der Notenkopf}
\label{notenkopf}
Der Notenkopf darf nicht zu groß oder zu klein sein. Er passt genau zwischen 
zwei Notenlinien. Eine Note wird nicht als Kreis gemalt. Der Kopf ist ein Oval 
(gestreckter Kreis) der schräg von links unten nach rechts oben verläuft (siehe 
\autoref{nkopf}).

\image{img/nkopfa.png}{der Notenkopf}{nkopf}{7}

Ein Notenkopf kann offen oder geschlossen sein. 
Ganze und halbe Noten haben einen offenen Kopf, alle andere Notenwerte werden
mit einem geschlossenen (schwarzen) Notenkopf gemalt (siehe \autoref{nkopf2}).

\image{img/nkopf2.png}{ein geschlossener und ein offener Notenkopf}{nkopf2}{4}


\section{Der Notenhals}
Der Notenhals hat in etwa die Länge von drei 
\comment{Hals $ \rightarrow $\\ drei Zwischenräume}Zwischenräumen.%
\footnote{Wenn die betreffende Note sich unter einem Balken befindet, kann es sein, dass der Hals verlängert oder verkürzt werden muss. Siehe dazu auch \autoref{balkensection}.}
Er kann sowohl nach 
oben als auch nach unten gezeichnet werden.
Notenhälse die nach oben zeigen, malt man rechts von der Note. Ein Hals der nach 
unten zeigt, kommt links von der Note (siehe \autoref{hals}).
\image{img/hals.png}{der Notenhals}{hals}{4}

Alle Noten bis auf die ganze Note haben einen Hals. In der 
Regel werden Noten, die über der dritte Linie stehen, mit dem 
Hals nach unten gezeichnet, die übrige Noten zeichnet
man mit dem Hals nach oben.

Bewegen Sie beim Malen der Notenhälse den Stift immer von oben 
nach unten.

\section{Hilfslinien}
Wenn Noten so tief (oder so hoch) sind, dass sie nicht mehr auf das
Notensystem (die fünf Linien) gezeichnet werden können, braucht man Hilfslinien
um die genaue Lage der Noten deutlich zu machen.

Eine Hilfslinie ist etwa so lange wie zwei Zwischenräume. Der Abstand
zwischen einzelnen Hilfslinien und zwischen Hilfslinie und Notensystem
beträgt genau einen Zwischenraum. Sie dazu \autoref{hlinie}.

\image{img/hlinie-1.png}{Eine halbe Note mit zwei Hilfslinien}{hlinie}{5}

\section{Fähnchen und Balken}
\label{balkensection}
Achtelnoten und Sechzehntelnoten werden mit Fähnchen dargestellt. 
Die Achtelnote bekommt 
ein Fähnchen, die Sechszehntelnote zwei. Siehe dazu \autoref{faenchen}.
Beachte, dass Fähnchen immer nach rechts gemalt werden, 
unabhängig davon, ob der Hals 
nach oben oder nach unten gezeichnet wurde.
Beachte, dass die Fähnchen etwa gleich lang sind wie der Notenhals.

\image{img/faehnchen.png}{Achtel- und Sechszehntelnote mit
    Fähnchen}{faenchen}{4}%

Die Fähnchen von mehreren Achtel- oder Sechzehntelnoten hintereinander können 
zusammengefügt werden zu einem Balken. Dadurch wird die Notation übersichtlicher.
Achtelnoten bekommen nur einen Balken, die Sechzehntelnoten einen Doppelbalken. 
Beide Sind dargestellt in \autoref{balken}.

\image{img/balken.png}{mehrere Achtel- und Sechzehntelnoten 
    mit Balken}{balken}{7}

Handelt es sich um Noten mit unterschiedlichen Tonhöhen, 
muss man gegebenenfalls die 
Länge der Hälse anpassen.
Die dritte Note in \autoref{balken2} bekommt einen längeren Hals, weil sonst die
Hälse der ersten oder der vierten Note zu kurz wären.
\image{img/balken2.png}{Noten unterschiedlicher Tonhöhe 
    mit Balken}{balken2}{3.5}

\section{Die Pausen}
Zu jedem Notenwert gibt es einen dazugehörigen Pausenwert. 
Die Pausen sind in \autoref{pausen} 
dargestellt.

\image{img/pausen.png}{die ganze Pause, halbe Pause, viertel Pause, 
    achtel Pause und die sechzehntel Pause}{pausen}{6}

\subsection{Ganze und halbe Pause}
Die ganze und die halbe Pause sind Rechtecken, die etwa 
zweidrittel eines Zwischenräumens dick
und etwas weniger als zwei Zwischenräume breit sind. 
Die ganze Pause hängt unterhalb der 
vierten Linie (von unten).\comment{ganze Pause hängt}
Die halbe Pause liegt auf der dritten Linie.\comment{halbe Pause liegt}
Siehe dazu \autoref{pause2}.
\image{img/pausen2c.png}{ungefähre Abmessungen der ganzen und 
    halben Pause}{pause2}{8}

\subsection{Die viertel Pause}
Die viertel Pause fängt oben zwischen der vierten und fünften Linie an, geht schräg nach rechts unten,
danach weiter lach links innerhalb vom dritten Zwischenraum (von unten), anschließend nach rechts,
nach unten
zum zweiten Zwischenraum, um da mit einem Bogen nach links abzuschließen. 
Siehe dazu \autoref{pause3}. 
\image{img/pausen3c.png}{die viertel Pause}{pause3}{5.5}

\subsection{Achtel- und Sechzehntelpausen}
In \autoref{pause4} werden die Achtel- und die Sechzehntelpause gezeigt.
\image{img/pausen4.png}{die Achtel- und die Sechzehntelpause}
{pause4}{5.5}

Man fängt im dritten Zwischenraum an, schreibt einen kleinen Bogen nach rechts,
anschließend eine gerade Linie nach unten links. Diese Linie sollte fast so lange sein
wie zwei Zwischenräume -- bei der Sechzehntelnote wie drei Zwischenräume. Siehe dazu
\autoref{pause5}
\image{img/pausen5c.png}{die Achtelpause mit Zeichenanleitung}{pause5}{5.5}


\section{Die Versetzungszeichen}
Die Versetzungszeichen (auch Vorzeichen) sind: das Kreuz, 
\comment{Kreuz $\sharp$\\Be $\flat$\\Auflösungszeichen $\natural$}
das Be und das Auflösungszeichen. Sie sind in 
\autoref{vzeichen} abgebildet. Beachte, dass beim Kreuz, die beiden Querstriche
sich etwa einen Zwischenraum auseinander befinden. 
Ebenso ist der "`Bauch"'
des Be etwa ein Zwischenraum groß, genauso wie die "`Raute"' beim 
Auflösungszeichen.
\image{img/vzeichen4.png}{das Kreuz, das Auflösungszeichen und das Be }
{vzeichen}{6}

Ein Versetzungszeichen befindet sich immer \emph{links} von der Note und auf der gleichen Höhe wie der Notenkopf. Siehe dazu \autoref{vzeichen2}
\image{img/vzeichen2.png}{Position der Versetzungszeichen}{vzeichen2}{5}

\chapter{Musiknotation}

\section{Ein ganzes Lied}
Wenn man ein Lied bzw. Musikstück in Notenschrift aufzeichnen soll, 
empfiehlt es sich eine bestimmte Reihenfolge einzuhalten. Man schreibt
Zeile für Zeile. Wie man die Zeilen im einzelnen erstellt wird in den
nächsten Abschnitten erläutert.


\section{Der Zeilenanfang}
Am Anfang der Zeile, ganz links kommt der Notenschlüssel. Dann schreibt man,
falls benötigt, die Vorzeichen (oder Versetzungszeichen) 
die die Tonart bestimmen. 
Danach erst kommt die Bezeichnung 
der Taktart (z.B. $ \begin{smallmatrix}3 \\ 4\end{smallmatrix} $ 
für einen 3-Vierteltakt). Siehe dazu 
\autoref{anfang}.

\image{img/anfang.png}{Anfang der ersten Zeile}{anfang}{4}

Notenschlüssel und Vorzeichen (falls vorhanden) werden auf jeder Zeile
wiederholt.\comment{Reiehnfolge:\\Notenschlüssel$ \rightarrow $\\
Vorzeichen$ \rightarrow $\\Taktart}
Die Taktart steht \emph{nur auf der ersten Zeile}. Falls ein 
Wechsel der Taktart stattfindet, schreibt man die neue Taktart
in den Takt wo der Wechsel statt 
findet -- nicht vorne beim Zeilenanfang. 

\section{Die Zeile}
Auf einer Zeile stehen mehrere Takte. Zwischen den Takten und am Ende der Zeile
malt man Orientierungsstriche -- die sogenannten Taktstriche.
\comment{Taktstriche dienen der Orientierung.}
Diese sind genau 4 Zwischenräume lang und gehen von der oberste bis 
zur unterste Linie (siehe \autoref{zeile}).

\image{img/zeile.png}{Erste Zeile mit Taktstrichen,
    noch ohne Noten}{zeile}{12}

Beim Schreiben muss man einschätzen,
wie viele Takte auf der Zeile passen. 
Drei bis vier Takte pro Zeile entspricht dem 
Normalfall, falls es jedoch viele halbe und ganze Noten gibt, 
kann es durchaus sein,
dass mehr als vier takte in der Zeile platz haben. 
Gibt es dagegen viele Sechzehntelnoten,
passen vielleicht nicht mehr als 2 Takte in der Zeile.

Nach dem Abschätzen malt man die Taktstriche. 
Es ist nicht nötig, 
alle Takte gleich breit 
zu machen. Takte mit weniger Noten darf man schmäler zeichnen. 

\section{der Takt}
Jetzt verteilt man die Noten und Pausen auf die einzelne Takte. 
In den Takt malt man erst 
nur die Notenköpfe und eventuelle Pausen hin. Dabei malt man
kleinere Notenwerte (Achtel, 
Sechzehntel) etwas enger 
zusammen als größere. Beachte natürlich, dass 
manche Notenköpfe offen, andere jedoch 
geschlossen sein müssen. Siehe dazu auch 
\autoref{notenkopf} auf Seite \pageref{notenkopf}.

\image{img/takt1-1.png}{1. Schritt: Noten ohne Hals}{takt1}{8}



\section{Hälse, Balken}
Wenn in einem Takt alle Notenköpfe eingeteilt wurden, 
entscheidet man sich bei den Achtel- 
und Sechzehntelnoten, ob man Balken oder Fähnchen 
zeichnen möchte (siehe nochmal 
\autoref{balkensection} auf Seite \pageref{balkensection}).

Die Notenhälse die nicht mit Balken verbunden werden sollen, 
kann man jetzt zeichnen (siehe \autoref{takt2}).

\image{img/takt2-1.png}{2. Schritt: einzelne Notenhälse einzeichnen}
{takt2}{8}

Falls es eine Notengruppe gibt, die unter einen (oder zwei)
Balken kommen sollen, malt man erst den Notenhals der linken Note
der Gruppe, danach den Notenhals der rechten Note der Gruppe. 
Ob die Hälse nach oben oder nach unten zeigen, 
wird von der Mehrheit der Noten bestimmt (siehe \autoref{takt3}).
 
\image{img/takt3-1.png}{3. Schritt: Achtelgruppe links und rechts}{takt3}{8}

Nachdem man den Balken gezogen hat (\autoref{takt4}), 
\image{img/takt4-1.png}{4. Schritt: Der Balken}{takt4}{8}

kann man die übrige Notenhälse dazu malen (\autoref{takt5}).
\image{img/takt5-1.png}{4. Schritt: Notenhälse Vervollständigen}{takt5}{8}

Diese Vorgehensweise (erst Köpfe, Pausen, danach Hälse und Balken)
\comment{Köpfe, Hälse, Balken (Hälse)} 
wiederholt man für die übrigen Takte der Zeile.

\section{Der Schluss}
Das Ende des Liedes bzw. des Musikstückes muss angegeben werden --
aus der Notation soll hervorgehen, dass das Lied zu Ende ist.
Dazu malt man als letzter Taktstrich einen Schlussstrich. 
\comment{Schlussstrich}
Dieser besteht
aus einem dünnen und einem dicken (etwa 3x so dick) Strich 
(siehe \autoref{schluss1}).

\image{img/schluss1-1.png}{Letzte Zeile mit Schlussstrich}{schluss1}{12}

\section{Ein Beispiel}
In \autoref{beispiel} sehen Sie eine Melodie, bestehend aus acht Takte,
auf zwei Zeilen untergebracht. Beachten Sie, dass die Taktart (hier
$\begin{smallmatrix}3\\4\end{smallmatrix}$) nur auf der erste Zeile steht.
Nicht alle Takte sind gleich breit. Die erste Zeile wird auf der rechten Seite
von einem Taktstrich abgeschlossen. Die letzte Zeile hat ganz rechts
einen Schlussstrich.
Versetzungszeichen -- hier zwei Kreuze -- stehen auf beiden Zeilen.


\image{img/beispiel-1.png}{Ein vollständiges Beispiel}{beispiel}{12}
\end{document}