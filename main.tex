\documentclass[10pt,a4paper,twoside]{report}
\usepackage[ngerman]{babel}
\usepackage{graphicx}
\usepackage{fancyhdr}
\usepackage{fontspec}
\usepackage{leading}
\usepackage[hidelinks]{hyperref}

\setmainfont[Ligatures=TeX]{Palatino Linotype}
\setsansfont[Ligatures=TeX]{Calibri}

%
%pagestyle
\pagestyle{fancy}
\renewcommand{\chaptermark}[1]{\markboth{#1}{} }
\renewcommand{\sectionmark}[1]{\markright{#1}{} }
\fancyhf{}
\fancyhead[LE,RO]{\thepage}
\fancyhead[LO,RE]{\nouppercase \leftmark}
\fancyheadoffset[LE,RO]{35mm}

\fancypagestyle{plain}{
	\fancyhf{}
	\fancyfoot[CE, CO]{\thepage}
	\renewcommand{\headrulewidth}{0pt}
}
%
% textwidth
\textwidth 13cm
\marginparwidth 3cm
\marginparsep 5mm
\parskip 0pt
\parindent 0pt

\renewcommand{\labelitemi}{$\cdot$}


%creates a marginpar with sans-serif font
\newcommand{\comment}[1]{
	\marginpar{
		\textsf{#1}
	}
}

% makes an image
% arg1: filename of image
% arg2: caption
% arg3: label (for \autoref{} )
% arg4: width in cm
\newcommand{\image}[4]{
	\begin{figure}[!ht]
		\centering
		\includegraphics[width=#4cm]{#1}
		\caption{#2}
		\label{#3}
	\end{figure}
}

%sets quote evironment
\newcommand{\myquote}[1]{
	\hspace{15mm}
	\parbox{100mm}{
		#1
	}
	\ \\
}

%
%author / title
\title{{\Huge Kinderpflege -- Musik}\\ \vspace{4mm}
    \small{ein Leitfaden\\
    für die Abschlussprüfung als andere Bewerber\\ 
    zum Erwerb des Berufsabschlusses\\ 
    als Staatlich geprüfter Kinderpfleger/\\ 
    Staatlich geprüfte Kinderpflegerin\\
    an der Fachakademie für Sozialpädagogik\\
    in das Fach Musische Gestaltung und Bewegungserziehung}
}
\author{Ivanildo Kowsoleea,\\Fachakademie für Sozialpädagogik Mühldorf}
\date{Mühldorf, im Januar 2014}

\hyphenation{Quin-te O-ber-quin-te So-zial-pä-da-go-gik Hol-län-di-schen vier Zwei-vier-tel-takt 
	Ak-kord Ak-kor-de be-fin-det ge-spielt}

%
% document
\begin{document}
\maketitle
\tableofcontents

\parskip 4pt
\leading{13.0pt}
\chapter{Einleitung}
\section{Warum dieses Skript}
Seit einigen Jahren müssen in Bayern diejenigen, die an einer Fachakademie für 
Sozialpädagogik das 
einjährige Sozialpädagogische Seminar (SPS) absolvieren 
(die sogenannte Quereinsteiger) und keine abgeschlossene Berufsausbildung
aufweisen, eine (externe) Prüfung zur KinderpflegerIn ablegen.%
\footnote{Siehe FakOSozPäd Anlage 3, Ziffer 10.1.2, Satz 2}

%: Erzieherpraktikanten, die unmittelbar in das zweite Jahr des Sozialpädagogischen Seminars eintreten und keine abgeschlossene Berufsausbildung in einem staatlich anerkannten Ausbildungsberuf aufweisen, haben sich einer Abschlussprüfung als andere Bewerber zum Erwerb des Berufsabschlusses als Staatlich geprüfter Kinderpfleger/ Staatlich geprüfte Kinderpflegerin an der Fachakademie für Sozialpädagogik zu unterziehen}

Da diese PraktikantInnen nur 0,5 Jahreswochenstunden Musikunterricht haben,
gibt es wenig Unterrichtszeit in der die KandidatInnen auf die
Prüfung vorbereitet werden können. Ich möchte hier die
nötigen musiktheoretischen Grundlagen darstellen
damit ebendiese Quereinsteiger 
sich selbstständig 
auf die Musikprüfung vorbereiten können.

\section{Zur Prüfung}
Während der Prüfung in das Fach Musik muss man nachweisen, 
dass man in der Lage ist,
sich ein Kinderlied anzueignen, 
dieses zum Klingen zu bringen (sowohl singend als spielend), 
zu begleiten und mit 
Kindern zu singen oder musizieren. Dazu sollen
Sprache (der Liedtext) und Bewegung (Tanz, Spiele) sinnvoll eingesetzt werden.

Damit der/die KandidatIn sich ein neues Lied aneignen kann, 
wird in diesem Skript zuerst das 
Notenlesen behandelt. Versetzungszeichen und Tonarten werden in aller Kürze 
behandelt um dem/der KandidatIn den Zugang zu Liedern in anderen Tonarten
als C-Dur zu ermöglichen. Auch wird kurz auf 
einfache Begleitungsarten eingegangen. 

%
% Kapitel: Tonsystem
%
\chapter{Tonsystem}
\section{Die Stammtöne}
In unserem Tonsystem gibt es sieben Töne. Von diesen sieben Tönen,
 die auch Stamm\-tö\-ne 
\comment{Stammtöne}
genannt werden, 
leitet man noch fünf andere Töne ab. Insgesamt besteht unser Tonsystem aus 
zwölf verschiedenen Tönen.

Die Stammtöne sind: \emph{c, d, e, f, g, a, h.} Siehe auch \autoref{stammtoene}
Diese Töne kann man beliebig -- höher oder tiefer -- wiederholen. Nach dem 
\emph{h} kommt wieder
\emph{c, d, e\dots} und vor dem c findet man \emph{\dots g, a, h.}

\image{lilypond/stammtoene.png}{Die Stammtöne in Notenschrift}{stammtoene}{6}


Wenn man die Töne von einem \emph{c} zum nächsten \emph{c} zählt 
(wobei man beide \emph{c's} mit zählt),
kommt man auf acht Töne. Dieser Abstand \emph{c --- c} nennt man 
\emph{Oktave}\footnote{von dem lateinischen \glqq 
	octavus\grqq -- der achte}.
\comment{Oktave}

Es gibt folglich mehrere \emph{c's}. 
Um anzudeuten welches \emph{c} man meint, muss man  
zusätzlich angeben in welcher Oktave sich der Ton 
befindet. Die Stammtöne in \autoref{stammtoene} gehören
zu der \emph{eingestrichenen Oktave} --- man sagt: 
\emph{c-eins, d-eins \dots} --- man schreibt
 $c^{1}, d^{1}, \dots$
 

Die nächsthöhere Oktave ist die \emph{zweigestrichene} Oktave: 
\emph{$c^{2}, d^{2}, \dots $}. Danach folgt die dreigestrichene Oktave
und so weiter. Die Oktave unter der eingestrichene heißt die 
\emph{kleine} Oktave (\emph{c, d, e, \dots}). Die Oktave darunter 
heißt \emph{große} Oktave (\emph{C, D, \dots}). Danach folgen die 
\emph{Kontraoktave} (\emph{$ C_{1}, D_{1},\dots $}) und die 
\emph{Subkontraoktave} 
(\emph{$C_{2}, D_{2}, \dots$}). Eine noch tiefere Oktave gibt es 
nicht, weil deren Töne so tief wären, dass man sie nicht mehr hören könnte.

\image{lilypond/oktaven.png}{ein Teil der kleinen Oktave,
    ein- und zweigestrichenen Oktave}{oktaven}{13}

Stimmen von Kindern im Kindergartenalter\footnote{vgl. Ernst, 2008 S. 13}
\comment{Umfang d. Kinderstimme}
bewegen sich in der Regel von 
$f^1$~bis~$d^2.$ 
Männer singen meistens in der \emph{kleinen Oktave}, Frauen in der Regel in
der \emph{eingestrichenen Oktave.}


\section{Versetzungszeichen}
Die Abstände zwischen den sieben Stammtönen sind nicht alle gleich. 
Meistens befindet sich zwischen
zwei Tönen ein \emph{Ganztonschritt}. Jedoch zwischen dem \emph{e} und 
\emph{f} und zwischen dem \emph{h} und
\emph{c} befindet sich ein Halbtonschritt. Die Abstände sind wie folgt:
\label{tonabstände}
\begin{center}
     \emph{$ c  -  d  -  e \cdot f  -  g  -  a  -  h \cdot c $}
\end{center}
wobei der Strich ($-$) einen Ganztonschritt zeigt und der Punkt 
($\cdot$) einen Halbtonschritt.

Der kleinster Abstand in unserem Tonsystem ist der Halbtonschritt.
\comment{Halbtonschritt\\Ganztonschritt} Zwei Halbtonschritte nacheinander 
ergeben einen Ganztonschritt. Deswegen gibt es zwischen dem 
\emph{c} und dem \emph{d} noch ein
Zwischenton. So ein Zwischenton gibt es auch zwischen
\emph{d} und \emph{e}, \emph{f} und \emph{g},
\emph{g} und \emph{a} und zwischen \emph{a} und \emph{h}, 
eben dort wo es in der Zeile mit den Abständen
einen Strich gibt. Ein Blick auf der Klaviertastatur 
zeigt uns wo es Zwischentöne gibt und wo nicht 
(siehe \autoref{klavier}). Die weiße Tasten entsprechen 
den Stammtönen und die schwarze Tasten den 
Zwischentönen.
\image{klavier.jpg}{Klaviertastatur}{klavier}{5}

Unsere Art Musik zu notieren gibt uns die Möglichkeit, 
einen Ton um einen Halbtonschritt 
höher oder Tiefer zu versetzen. Dazu schreibt man ein Versetzungszeichen 
(ein \emph{Kreuz} oder ein \emph{b}) vor einer Note.
\comment{Kreuz: $\sharp$\\b: $\flat$}
Ein \emph{Kreuz} macht eine Note um einen Halbtonschritt höher. 
Ein \emph{b} macht eine 
Note um einen Halbtonschritt tiefer. Ein Kreuz sieht so 
aus: $\sharp$, und ein b: $\flat$.

Um eine Versetzung auszudrucken ändert man den Namen des Tons. 
Eine Erhöhung deutet man an mit der
Silbe -is, die man den Notennamen anhängt. 
Ein "`tiefergelegter"' Ton bekommt die Silbe 
-es angehängt. Es gibt einige Ausnahmen. Siehe dazu die 
fettgedruckten Einträge in der \autoref{notennamen}.
\begin{table}[ht]
    \centering
    \begin{tabular}{|c|c|c|}
    \hline 
    $\flat$ & Stammton & $\sharp$ \\ 
    \hline \hline
    ces & c & cis \\ 
    \hline 
    des & d & dis \\ 
    \hline 
    \textbf{es} & e & eis \\ 
    \hline 
    fes & f & fis \\ 
    \hline 
    ges & g & gis \\ 
    \hline 
    \textbf{as} & a & ais \\ 
    \hline 
    \textbf{b(!)} & h & his \\ 
    \hline 
    \end{tabular} 
    \caption{Notennamen der Stammtönen und der Zwischentönen}
    \label{notennamen}
\end{table}

Beachte den Notennamen in der letzten Zeile, in der linke Spalte. 
Diese Note \emph{(b)} hätte eigentlich \emph{hes} heißen müssen.
\comment{hes $\rightarrow$ b}
Es ist jedoch eine Ausnahme, deren Ursprung
historische Gründe hat.

Beachte, dass man das Versetzungszeichen \emph{links vor} der Note 
auf gleicher Höhe schreibt. 
Siehe auch \autoref{verszeichen} 
\image{lilypond/versetzung.png}{Notenschrift mit
     Versetzungszeichen}{verszeichen}{5}

Die Note zwischen \emph{c} und \emph{d} kann man erreichen, indem man das 
\emph{c} zum \emph{cis} erhöht.
Diese Note lässt sich jedoch ebenso erreichen, indem man das 
\emph{d} zum \emph{des} heruntersetzt. Dieser Zwischenton
hat daher zwei Namen, 
\comment{cis = des}
abhängig davon, mit welcher Stammton man anfängt
(\emph{c} oder \emph{d}). Beide Bezeichnungen sind korrekt.

Die sieben Stammtöne ergeben mit den fünf Zwischentöne insgesamt zwölf Töne pro Oktave. Diese Töne sind in \autoref{toene} nochmal dargestellt.
\begin{table}[ht]
	\centering
	\begin{tabular}{|p{5mm}|p{5mm}|p{5mm}|p{5mm}|p{5mm}|p{5mm}|p{5mm}|p{5mm}|p{5mm}|p{5mm}|p{5mm}|p{5mm}|}
	\hline 
	c\newline {\tiny (his)} & 
	cis\newline des & 
	d & 
	dis \newline es 
	& e\newline {\tiny (fes)} & 
	f\newline {\tiny (eis)} & 
	fis\newline ges & 
	g & 
	gis\newline as 
	& 
	a & 
	ais\newline b & 
	h \newline {\tiny (ces)} \\ 
	\hline 
	\end{tabular} 
	\caption{Die Töne unseres Tonsystems}
	\label{toene}
\end{table}
Die kleine, eingeklammerte Notennamen (z.B. \emph{his}) werden in der 
Regel eher selten benutzt, auch wenn sie theoretisch korrekt wären.

\section{Tonarten}
Am Anfang von \autoref{tonabstände} auf \autopageref{tonabstände} wurden die Stammtöne
mit den Tonabständen dargestellt. Die Abstände zwischen den Tönen sind nicht immer gleich.
Wenn ich jetzt eine Tonreihe mit einem anderen Ton als \emph{c} anfangen muss -- weil ich 
ein Lied höher oder tiefer singen möchte -- muss ich natürlich diese unterschiedliche
Tonabstände mit einbeziehen.

Als Beispiel möchte ich eine neue Reihe (eine Tonleiter) konstruieren, die mit dem Ton 
\emph{d} anfängt. Wenn ich die Stammtöne neben einander schreibe entsteht:
\begin{center}
	\emph{$d - e \cdot f - g - a - h \cdot c - d$}
\end{center}
Wir wissen aber -- siehe nochmal die Tonabstände auf \autopageref{tonabstände} --
das die zwei Stellen mit den Halbtonschritten sich zwischen \emph{e} und \emph{f}
befinden und zwischen \emph{h} und \emph{c}.

\chapter{Rhythmus}
\section{Notenwerte}
Es reicht nicht, zu wissen welcher Ton man spielen (oder singen) soll. Man
muss auch wissen, wann der nächste Ton folgt. Es gibt Töne die 
länger oder kürzer
dauern als andere. Wie lange ein Ton dauert 
wird von seinem Wert (der Notenwert)\comment{Notenwert $\rightarrow$ Dauer}
bestimmt. Es gibt folgende Notenwerte%
\footnote{Notenwerte 
kürzer als die 32-tel Note gibt es zwar, sind jedoch 
nicht Gegenstand der Prüfung.}
(siehe \autoref{notenwerte}).
\begin{table}[!ht]
	\centering
	\begin{tabular}{|c|c|}
	\hline \textbf{Name} & \textbf{Wert} \\
	\hline ganze Note & $1$ \\ 
	\hline halbe Note & $\frac{1}{2}$ \\ 
	\hline Viertelnote & $\frac{1}{4}$ \\ 
	\hline Achtelnote & $ \frac{1}{8} $ \\ 
	\hline Sechszehntelnote & $ \frac{1}{16} $ \\ 
	\hline 32-tel Note & $ \frac{1}{32} $ \\ 
	\hline 
	\end{tabular} 
	\caption{Die Notenwerte}
	\label{notenwerte}
\end{table}
Diese Werte sagen nichts darüber aus, wie lange (in Sekunden) eine Note dauert. 
Es ist lediglich festgelegt, dass z.B. eine ganze Note 
viermal so lange dauert wie eine
Viertelnote. Die Zeitspanne für einen Notenwert kann der Spieler selber 
festlegen. Man kann ein Lied schneller oder langsamer anstimmen.

Die Notenzeichen zu den Notenwerte sind in \autoref{notenzeichen} dargestellt.%
\image{lilypond/notenwerte.png}{die ganze Note, halbe Note, Viertelnote,
Achtelnote, Sechszehntelnote und die 32-tel Note}{notenzeichen}{8}
Die Noten mit einem Wert kürzer als die Viertelnote (das sind die mit Fähnchen) 
werden, falls es mehrere hintereinander gibt, oft zusammengefasst. 
Die Fähnchen werden 
durch einen Balken ersetzt so wie in \autoref{balken}.
\image{lilypond/balken.png}{mehrere Achtel- und Sechszehntelnote zusammengefasst}{balken}{6.5}

\section{Metrum und Taktart}
Um die Geschwindigkeit eines Musikstückes bestimmen zu können, benutzt man
den Grundschlag. Das ist ein regelmäßig wiederkehrender Impuls, der zur
Orientierung dient. \comment{Metrum}Dieser regelmäßig 
wiederkehrender Impuls nennt man auch \emph{Metrum.}

Die Impulse des Grundschlags sind in Gruppen unterteilt. So eine Gruppe heißt 
\emph{Takt.}\comment{Takt}Pro Takt kann es zwischen zwei und neun
Grundschläge geben. 
In der Regel ändert sich die Anzahl der Impulse pro Takt während eines
Liedes nicht (Taktwechsel), es gibt jedoch Ausnahmen. 


Die Anzahl der Grundschläge pro Takt wird von der \emph{Taktart}
bestimmt.\comment{Taktart}Diese bestimmt auch welcher Notenwert
mit dem Grundschlag übereinstimmt. Die Taktart wird mit zwei 
übereinander gesetzte Zahlen angegeben. Die obere Zahl bestimmt die Anzahl
der Grundschläge pro Takt -- die untere Zahl zeigt welcher Notenwert mit dem
Grundschlag übereinstimmt.


Einige Beispiele:
\parskip 0pt
\begin{itemize}
\item In einem vierviertel-Takt $ \left( \begin{array}{c}\textbf{4} \\ \textbf{4} \end{array}\right)  $
gibt es vier Schläge pro Takt; die Viertelnote dauert ein Grundschlag.
\item In einem sechsachtel-Takt $ \left( \begin{array}{c}\textbf{6} \\ \textbf{8} \end{array}\right)  $
gibt es sechs Schläge pro Takt wobei die Achtelnote ein Grundschlag dauert.
\end{itemize}
\parskip 3pt

In Notenschrift werden die Noten die zu einem Takt gehören zwischen senkrechten 
Strichen -- die Taktstriche -- gestellt.
Der erster Impuls des Taktes bzw. die erste Note nach einem Taktstrich, wird
leicht betont, damit hörbar wird, wann der Takt anfängt. Man zählt daher (z.B. 
Viervierteltakt):

\hspace{2cm}\textbf{eins} -- zwei -- drei --vier -- \textbf{eins} \dots

und spielt/singt die erste Viertelnote im Takt etwas betont. In 
\autoref{viervier}sind
\image{lilypond/viervier.png}{einige Viertelnoten in
    Viervierteltakt}{viervier}{6}
die Noten die mit einer "`1"' markiert sind, werden betont. Dadurch, dass diese
Betonung alle vier Schläge wiederkehrt, kann man den Viervierteltakt 
hören. Die  
\comment{Hauptakzent}Betontung auf dem erste Schlag des 
Taktes nennt man auch \emph{Hauptakzent.}

\section{Rhythmus}
Nicht alle Töne eines Liedes sind gleich lang. Dieser ungleiche Verteilung der 
Tonlängen nennt man Rhythmus.\comment{Rhythmus}
Betrachten wir beispielsweise das Lied "`Hänsel
und Gretel"' dessen erste Zeile in \autoref{haensel} zu sehen ist. %
Es gibt einen Viervierteltakt,
die erste Note jedoch (eine halbe Note) dauert zweimal so lange wie die zwei
Viertelnoten die danach folgen. 
\image{lilypond/haensel.png}{Hänsel und Gretel}{haensel}{10}%

Das Metrum ist ein regelmäßiger Ablauf von Taktschlägen,
und das kann man als Hilfe benutzen um einen Rhythmus wie diesen auszuführen.
Dazu zählt man leise mit (\textbf{eins,} zwei, drei vier, \textbf{eins,} \dots),
und singt/spielt die Noten wie folgt (siehe \autoref{haensel2}):
\image{lilypond/haensel2.png}{Metrum in Kombination mit Rhythmus}{haensel2}{10}

\myquote{
	Hän- {\footnotesize (\textbf{eins,} zwei)} sel {\footnotesize (drei)} und 
	{\footnotesize (vier)} Gre- {\footnotesize (\textbf{eins,} zwei)}
	tel {\footnotesize (drei)} ver- {\footnotesize (vier)} lie- 
	{\footnotesize (\textbf{eins})} fen {\footnotesize (zwei)} sich 
	{\footnotesize (drei)} im {\footnotesize (vier)}
	Wald {\footnotesize (\textbf{eins,} zwei, drei, vier)}
}

So kann man einigermaßen sicher sein, dass die Notenwerte korrekt ausgeführt 
werden.

Falls es im Lied kleinere (kürzere) Notenwerte gibt als die, welche im Grundschlag 
angegeben sind, muss man zwischen den Schlägen die Silbe \glqq und\grqq\ denken. 
Beim Lied \glqq Grün, grün, grün sind \textellipsis\grqq\ (siehe \autoref{gruen})
\image{lilypond/gruen.png}{Lied mit Achtelnoten}{gruen}{10}
gibt es im dritten Takt vier Achtelnoten. Den Grundschlag bilden hier die 
Viertelnoten, weil dieses Lied im Zweivierteltakt steht. Die Achtelnoten sind 
demzufolge schneller als der Grundschlag.
\image{lilypond/gruen2.png}{Zählweise bei Achtelnoten}{gruen2}{11}
Deshalb zählt man im dritten Takt (siehe \autoref{gruen2}):

\myquote{\textbf{eins} -- und -- zwei -- und\dots}

Wenn man den Grundschlag in vier Teilen muss, wie es z.B. bei Sechzehntelnoten der
Fall sein kann, zählt man: 

\myquote{\textbf{eins} -- e -- un -- te -- zwei\dots}

\image{lilypond/sechzehn.png}{Zählweise bei Sechzehntelnoten}{sechzehn}{7}

\section{Haltebögen und Punktierte Noten}
Eine Note die durch einen Haltebogen mit einer zweiten Note verbunden ist,
wird langer gespielt/gesungen, und zwar so lange wie beide Notenwerte zusammen.
Der Haltebogen\comment{Haltebogen} darf man nicht mit dem Bindebogen verwechseln.
Man spricht von einem Haltebogen, wenn zwei Noten derselben Tonhöhe mit einander
verbunden werden. Wenn beide Noten unterschiedliche Tonhöhen aufweisen, spricht man
von einem Bindebogen. Dabei sollten die Noten verbunden gesungen werden -- 
z.B. auf einer Textsilbe. Siehe dazu \autoref{bogen}.
\image{lilypond/bogen.png}{ein Haltebogen und ein Bindebogen}{bogen}{6}

Der erste bogen verbindet zweimal ein $c^2$. Es ist demnach ein Haltebogen, und man
spielt \emph{einen} Ton der \emph{drei} Schläge dauert. Das erste $c^2$ dauert zwei 
Schläge, das zweite $c^2$ ist eine Viertelnote und dauert einen Schlag. Beide Noten
zusammen dauern deshalb ($2 + 1 = 3$) drei Schläge.

Jetzt ist auch klar, wie man eine Note mit drei Schläge erstellen kann. Das ist 
nicht möglich ohne den Einsatz des Haltebogens. Weil diese Situation
\comment{Punkt verlängert um die Hälfte} (drei Schläge)
öfter vorkommt, gibt es dafür eine vereinfachte Schreibweise; hinter der ersten (halbe)
Note schreibt man einen Punkt. Siehe \autoref{bogen2}.
\image{lilypond/bogen2.png}{eine punktierte halbe Note}{bogen2}{6}

Der Punkt verlängert eine Note um die Hälfte ihres Wertes. Das geht auch bei anderen 
Notenwerten, z.B. Eine punktierte Viertelnote in einem Dreivierteltakt dauert einen und 
ein halben Schlag. Siehe dazu \autoref{schnee}.
\image{lilypond/schnee.png}{der Anfang von \glqq Leise rieselt der Schnee\grqq }{schnee}{7}

Wenn man den zweiten Takt zählen möchte, zählt man

\myquote{
	rie- {\footnotesize (\textbf{eins,} zwei)} selt {\footnotesize (und)} der 
	{\footnotesize (drei)} Schnee {\footnotesize (\textbf{eins,} zwei, drei)}
	\dots
}

Beachte übrigens den Haltebogen der die beiden letzten Takte verbindet. So wird
eine Note geschrieben die \emph{sechs} Schläge dauert. Das wäre in einen Dreivierteltakt sonst nicht möglich!

\section{Ein unbekanntes Lied}
Wenn man ein neues Lied einüben möchte, ist es leichter sich erst auf dem Rhythmus zu 
konzentrieren. Dabei ignoriert man erstmal die Tonhöhen und spricht zunächst erst 
den Text. In den Takten wo man den Rhythmus nicht kennt, spricht man die Grundschläge
so wie es gerade beschrieben wurde (z.B. \textbf{eins} -- zwei - und - drei -- vier).
Erst wenn man den Text mit dem richtigen Rhythmus sprechen kann, \glqq sucht\grqq\ man 
die richtige Tonhöhen dazu.

\chapter{Begleitung}
\section{Begleitung mit einem Bordun}
Ein Bordun\footnote{von \textbf{Bourdon} (frz.) = Hummel, eine Anspielung auf das Summen einer Hummel}
ist ein andauernder, gleichbleibender, summender und meist tiefer Ton, \comment{Bordun = 
ein tiefes Summen} der als Begleitung einer Melodie gespielt oder gesungen wird.\footnote{vgl. Willemze Theo: 
Algemene Muziekleer §1051: Bourdon kommt von burden (engl.) = Schwere; 
Borduntöne sind schwere/tiefe Töne}
Dieser Ton lässt man während des ganzen Lieds klingen. Wenn der Ton auf den zur Verfügung stehenden Instrumenten
nicht ausgehalten werden kann (z.B. auf Stabspielen) wiederholt man den so oft wie nötig, mindestens 
einmal im jeden Takt.

Lieder die nur wenige Begleitakkorde benötigen, eignen sich für eine Begleitung mit einem Bordun. 
Normalerweise nimmt man als Begleitton den Grundton der Tonart. Wenn das zu begleitende Lied z.B. 
in F-Dur steht, nimmt man ein (tiefes) F. Oft findet man in den Noten der Lieder die Begleitakkorde.
In dem Fall sieht man, ob es viele Akkordwechsel gibt, und welcher Akkord am meisten genutzt wird. 
So kann man den Grundton -- den Bordunton -- leicht ermitteln.

Oft wird der Bordun mit einer Oberquinte\footnote{Quinte von quintus (lat.) = der Fünfte.
Die Oberquinte
liegt fünf Stammtöne -- oder sieben Halbtonschritte -- über den Grundton.} \comment{Oberquinte} bereichert.
Beide Töne (Bordun und Oberquinte) werden zusammen
-- eventuell auf verschiedenen Instrumenten -- gespielt. 

\section{Begleitung mit einem Ostinato}
\glqq Ostinato\footnote{Laut Duden sowohl \emph{der} Ostinato als auch \emph{das} Ostinato}\grqq\ kommt aus dem Italienischen und heißt \emph{hartnäckig}.\comment{ostinato =\\hardnäckig}
In der Musik wird damit ein immer wiederkehrende Tonfolge angedeutet. Diese Tonfolge kann aus sehr vielen
Tönen bestehen, für die Begleitung von Kinderlieder beschränkt man sich in der Regel auf kürze Tonfolgen
von bis zu vier Tönen.

Lieder die Sich für einen Bordun eignen, lassen sich auch mit einem Ostinato begleiten. Ferner kann man
Lieder die einem Regelmäßigen Akkordwechsel aufweisen mit einem Ostinato so begleiten, das die Tonfolge
genau dem Akkordwechsel folgt. Somit sind die Möglichkeiten beim Ostinato etwas größer als beim Bordun.

\section{Begleitung mit Grundtönen}
Bei den Liedern, wo die Akkorde schon vorgegeben sind, kann man die Grundtöne der Akkorde von (tiefen) 
Instrumenten spielen lassen. Die Grundtöne lassen sich leicht ermitteln. Der Akkordbuchstabe gibt den 
Grundton an. So ist z.B. bei dem Akkord $Am$ (a-Moll) der Grundton: $A$.

Ob man diese Begleitungsart in einem Kindergarten von Kindern bewältigen lassen kann, ist fraglich. 
Erwachsene Mitspieler (KollegInnen) könnten das ohne weiteres. Falls man nicht über einen Bass-Xylofon 
oder Ähnliches verfügt,
eignet sich eine Gitarre oder Keyboard eben so gut. Vor allem wenn die Akkorde auf der
Gitarre schwierig zu spielen sind 
(z.B. F) ist eine Begleitung mit (nur) Basstönen/Grundtönen eine gute Alternative.

\chapter{Musizieren mit Kindern}
\section{Warum musizieren?}
Musik kann in unseren Alltag unterschiedliche Funktionen haben. Das Erleben von Musik kann uns 
Menschen positiv beeinflussen. Es scheint deshalb wünschenswert, Kinder möglichst früh mit 
Musik in Kontakt zu bringen. Dabei muss einer der Ziele sein, dem Kind zu befähigen, die positive
Einflüsse für sich zu nutzen. Dazu muss die Musik sowohl passiv als auch aktiv
erlebt werden.

In der Soziologie unterscheidet man verschiedene Funktionen, die von Musik ausgeübt werden 
können.\footnote{vgl. Rösing H.: Kulturpsychologische Aspekte; aus: Bruhn, Oerter, Rösing (Hg.) Musikpsychologie --
ein Handbuch, S. 77}
Einige für die Prüfung relevante Funktionen sind:
\begin{itemize}
	\item Festlichkeitsfunktionen: Musik als Rahmen für Besondere Anlässe oder Feierlichkeiten
	\item Funktionen der Bewegungsaktivierung: Tanzmusik, Marsch- oder Parademusik
	\item gemeinschaftsbindende, gruppenstabilisierende Funktionen: Musik als Identifikationsmerkmal
	\item erzieherische Funktionen: Musik als Mittel zur Bildung
	\item Kontaktfunktionen: als nonverbales Medium zu Kontaktaufnahme 
\end{itemize}
Dieser Vielfalt an Funktionen macht einerseits klar, dass auf Musik nicht verzichtet werden kann, andererseits
werden einige Anlässe genannt, die sich zum Einsatz von Musik eignen. So kann man z.B. Bewegungsangebote
mit Musik untermalen. Gemeinsames singen am Morgen kann die Gemeinschaft stärken. Dass Musik bei den
verschiedenen Festen -- religiöse oder weltliche Anlässe -- eine wichtige Rolle spielt, muss man
hier nicht weiter betonen. Die erzieherische Funktionen können sein:
\begin{itemize}
	\item das Schulen der auditiven Wahrnehmung
	\item das Verständnis der Sprache fördern 
	\item die Förderung der Sprachmotorik
	\item die Entwicklung der Grobmotorik unterstützen
\end{itemize}


\section{Methodische Aspekte}
Bei der Planung eines Musikalischen Angebots muss man sich immer vor Augen halten,
dass Kinder grundsätzlich spielen möchten.\comment{Spielpartner}
Deshalb muss der Erzieher oder Kinderpfleger sich in erste Linie als 
Spielpartner anbieten.\footnote{vgl. den Beitrag \glqq Von der 
	Skepsis zur Überzeugung\grqq\ von Küspert I. in Ribke u. Dartsch (Hg.) Facetten Elementare
	Musikpädagogik, S. 52ff} Jedes Spiel braucht ein Thema\comment{Spiel -- Thema}
das für diese 
Altersstufe interessant ist. Das Singen, Trommeln oder Tanzen kann so auf
natürliche Weise mit einbezogen werden, ohne das bei den Kindern das Gefühl
entsteht das sie etwas tun (im Sinne von arbeiten) \emph{müssen.}


Kinder brauchen Wiederholung.\comment{Wiederholung} Man kann ein Lied nicht oft
genug singen. 
Bei der Planung muss man das berücksichtigen, und ausreichend Zeit dafür nehmen.
Meiner Meinung nach kann man durchaus weniger machen (z.B. ein kürzeres Lied, 
nur wenige Tanzschritte), und diese Einheit dafür öfter auftauchen lassen.

Erwachsene haben in der Regel eine ausgeprägte Meinung darüber, wie
Musik zu sein hat. Bei Kindern ist das noch nicht der Fall, und sie\comment{Experimente} 
werden sich deshalb eher auf Experimente einlassen. Das Bietet gute 
Voraussetzungen für das Improvisieren, \comment{Improvisation} das spontane
-- ohne Vorgaben wie z.B. eine Melodie -- Spielen oder Singen.
Damit eine Improvisation gelingen kann müssen einige Bedingungen geschaffen 
werden:\footnote{nach Greiner J., \glqq Holzmusik -- Ein Beispiel aus der Elementaren
	Musikpädagogik mit Vorschulkindern\grqq; Beitrag in Ribke u. Dartsch (Hg.) Facetten
	Elementare Musikpädagogik, S. 57ff}

\myquote{-Vertrautheit,}

\myquote{-etwas das die Fantasie anregt,}

\myquote{-Verabredungen um das Geschehen zu strukturieren,}

\myquote{-ausreichende akustische Sensibilisierung,}

\myquote{-Handhabung der Instrumente, und}

\myquote{-es gibt kein \glqq Richtig\grqq\ oder \glqq Falsch\grqq\ }

Um diese Bedingungen zu schaffen braucht man natürlich einige Stunden zur Vorbereitung.
Die Instrumente sollten vorher ausprobiert sein (Handhabung), die Spielregeln
(Verabredungen: wer darf wann spielen) kann man in früheren Stunden einführen
und üben, und damit Kinder aufeinander hören lernen, braucht es Zeit. Nach
ausreichende Vorbereitung allerdings, kann eine gelungene Improvisation zu 
einem intensiven Erfolgserlebnis führen.

Folgende Instrumente\comment{Instrumente} sind für Kinder im Kindergartenalter 
gut geeignet:\footnote{
	Fischer R: Singen Bewegen Sprechen -- Musik machen in Kita und Krippe, S 37}
Klanghölzer, Rasseln, Handtrommeln, Triangel, Becken (verschiedene Größen),
Guiros, Woodblocks, Schellenreifen, Altxylofon (zum Vorspielen oder Mitspielen) oder
Glockenspiel (helle Töne). Suchen Sie nach
Bildern dieser Instrumente, damit Sie diese benennen können.

\chapter{Die Prüfung}
\section{Vorbereitung}
Wählen Sie drei Lieder aus. Diese Lieder müssen für Kinder im 
Kindergarten- oder Grundschulalter angemessen sein. Überlegen Sie, was sie mit diesen Liedern
mit einer Gruppe von Kindern im dem Alter machen können. Überlegen Sie
sich mögliche Rahmenhandlungen oder Geschichten. Wenn es angemessen 
erscheint, überlegen Sie sich Bewegungen, Gesten oder Tanzschritte.

Bedenken Sie, was Sie mit diesen Liedern Fördern könnten (z.B. Sprachverständnis, 
Grobmotorik, Sprachmotorik\ldots). Überlegen Sie, wie Sie diese Förderung noch 
verstärken könnten, z.B. indem Sie bei Sprachmotorik mehr auf die Sprache eingehen,
oder extra Sprachübungen durchführen. Gibt es eventuell Angebote, die Sie vorher
durchführen müssen? Denken Sie dabei an einer Geschichte die man vielleicht vorher
erzählen möchten. Möchten Sie Tanzschritte im Vorfeld einüben?

Üben Sie ihre Lieder. Sie müssen diese singen und spielen können. Es ist gleich, welches
Instrument Sie verwenden möchten. Achten Sie bitte auf eine für Kinder geeignete Tonhöhe.
Überlegen Sie eine passende Begleitung. Besonders gut wäre es, wenn diese Begleitung auch von
Kindern gespielt werden könnte.

\section{Die mündliche Prüfung}
Während der Prüfung stellen Sie den Prüfern ihre Lieder vor. Daraufhin wählen die Prüfer
ein Lied aus. Anschließend stellen Sie Ihre Überlegungen zu dem Lied vor. Rechnen Sie 
während Ihre Vorstellung mit ergänzende Fragen der Prüfer. Nach Abschluss folgen noch einige 
Fragen allgemeiner Natur.

Nach etwa 30 Minuten werden Sie gebeten, den Raum für kurze Zeit zu verlassen. Die Prüfer
werden sich beraten und Sie nach einigen Minuten nochmal hereinbitten, damit Ihnen das 
Ergebnis mitgeteilt werden kann. Jetzt können Sie sich, nach bestandener Prüfung, entspannen.
Wir wünschen Ihnen viel Erfolg.

%\bibliographystyle{plain}
%\bibliography{main}

\chapter*{Literaturverzeichnis}
\begin{itemize}
\item FakOSozPäd. Anlage 3, 10.2.1 Satz 2 lautet:
\glqq Erzieherpraktikanten, die unmittelbar in das zweite Jahr des Sozialpädagogischen Seminars eintreten und keine abgeschlossene Berufsausbildung in einem staatlich anerkannten Ausbildungsberuf aufweisen, \emph{haben sich einer Ab\-schluss\-prüfung} als andere Bewerber zum Erwerb des Berufsabschlusses als Staatlich ge\-prüf\-ter Kinderpfleger/ Staatlich geprüfte Kinderpflegerin an der Fach\-aka\-demie für Sozialpädagogik \emph{zu unter\-zie\-hen.\grqq}

% § 37 Abs. 1 Satz 2, Nrn. 10.1.1 bis 10.1.4 sowie die §§ 44, 46 bis 48, 50 Abs. 2 Sätze 2 bis 5,§§ 51 und 51a Abs. 1 BFSOHwKiSo gelten entsprechend.
\item Ernst, Manfred: Praxis Singen mit Kindern, Lieder vermitteln, begleiten, dirigieren. Rum/Innsbruck, Helbling 2008

\item Willemze Theo: Algemene Muziekleer, neunte verbesserte Auflage 1984, Uitgeverij Het Spectrum, Utrecht/Antwerpen (Übersetzungen aus dem Holländischen stammen vom Verfasser dieses Skriptes)

\item Bruhn H, Oerter R. und Rösig H. (Hg.): Musikpsychologie -- ein Handbuch, Rowolt Taschenbuch 
Verlag, Reinbeck bei Hamburg, Mai 1994

\item Fischer R: Singen Bewegen Sprechen -- Musik machen in Kita und Krippe, Schott Verlag
\end{itemize}
\newpage
\ 
\end{document}